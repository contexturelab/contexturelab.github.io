%-------------------------
% Resume in Latex
% Author : Jake Gutierrez
% Based off of: https://github.com/sb2nov/resume
% License : MIT
%------------------------

\documentclass[letterpaper,12pt]{article}

\include{preamble}
%-------------------------------------------
%%%%%%  RESUME STARTS HERE  %%%%%%%%%%%%%%%%%%%%%%%%%%%%

\begin{document}
\normalem
%----------HEADING----------
% \begin{tabular*}{\textwidth}{l@{\extracolsep{\fill}}r}
%   \textbf{\href{http://sourabhbajaj.com/}{\Large Sourabh Bajaj}} & Email : \href{mailto:sourabh@sourabhbajaj.com}{sourabh@sourabhbajaj.com}\\
%   \href{http://sourabhbajaj.com/}{http://www.sourabhbajaj.com} & Mobile : +1-123-456-7890 \\
% \end{tabular*}

\begin{center}
    \textbf{\Large \scshape Sadamori Kojaku} \\ \vspace{1pt}
    Assistant Professor at School of Systems Science and Industrial Engineering \\
    Thomas J. Watson College of Engineering and Applied Science, Binghamton University \\
    EB-J19, 4400 Vestal Pkwy E, Binghamton, NY 13902 \\
    \href{mailto:skojaku@binghamton.edu}{\underline{skojaku@binghamton.edu}} $|$
    \href{812-646-8447}{\underline{812-646-8447}} $|$
    \href{https://skojaku.github.io/}{\underline{skojaku.github.io}}
\end{center}



%-----------EDUCATION-----------
\section{Education}
  \resumeSubHeadingListStart
    \resumeSubheading
      {Ph.D. Computer Science}{Sep. 2013 -- Sep. 2015}{Hokkaido University, Japan (Thesis supervisor: Prof. Mineichi Kudo)}{}
    \resumeSubheading
      {M.S. System Engineering}{Apr. 2010 -- Mar. 2012}{Hokkaido University, Japan (Thesis supervisor: Prof. Hajime Igarashi)}{}
    \resumeSubheading
      {B.S. System Engineering}{Apr. 2007 -- Mar. 2010}{Hokkaido University, Japan (Thesis supervisor: Prof. Hajime Igarashi)}{}
\resumeSubHeadingListEnd
%-----------EXPERIENCE-----------
\section{Experience}
  \resumeSubHeadingListStart
    \resumeSubheading
      {Assistant Professor}{Aug. 2023 -- Present}
      {Systems Science and Industrial Engineering, Binghamton University}{Binghamton, USA}
      \resumeItemListStart
        \resumeItem{Conducting research in the field of complex systems and network science}
        \resumeItem{Supervising Three Ph.D. students in their dissertations and research projects}
        \resumeItem{Teaching two graduate courses in Systems Science and Industrial Engineering}
      \resumeItemListEnd

    \resumeSubheading
      {Postdoctoral Research Fellow}{Feb. 2020 -- Jul. 2023}
      {School of Informatics, Computing, and Engineering, Indiana University}{Bloomington, USA}
      \resumeItemListStart
        \resumeItem{Developed algorithms for analyzing large-scale data on scholarly data}
        \resumeItem{Mentored one Master and several Ph.D. students in their research projects}
        \resumeItem{Teaching a graduate course on Data Visualization}
      \resumeItemListEnd

    \resumeSubheading
      {Specially Appointed Professor (Postdoc)}{Apr. 2019 -- Jan. 2020}
      {Research Institute for Economics and Business Administration, Kobe University}{Kobe, Japan}
      \resumeItemListStart
        \resumeItem{Developed a search engine for finding relevant reviewers for a paper (patented)}
        \resumeItem{Mentored one Ph.D student in developing their research proposals and methodologies}
      \resumeItemListEnd

    \resumeSubheading
      {Research Associate}{Apr. 2016 -- Mar. 2019}
      {Department of Engineering Mathematics, University of Bristol}{Bristol, UK}
      \resumeItemListStart
        \resumeItem{Conducted research on complex networks}
      \resumeItemListEnd

% -----------Multiple Positions Heading-----------
%    \resumeSubSubheading
%     {Software Engineer I}{Oct 2014 - Sep 2016}
%     \resumeItemListStart
%        \resumeItem{Apache Beam}
%          {Apache Beam is a unified model for defining both batch and streaming data-parallel processing pipelines}
%     \resumeItemListEnd
%    \resumeSubHeadingListEnd
%-------------------------------------------

\resumeSubHeadingListEnd


%-----------PROJECTS-----------
\section{Papers under review}
(BU students underlined; my name in italics)
\begin{itemize}[leftmargin=0.6in, label={\paperitem}]
  \item Munjung Kim, \emph{Sadamori Kojaku}, Yong-Yeol Ahn. Uncovering simultaneous breakthroughs with a robust measure of disruptiveness. arXiv preprint arXiv:2502.16845, 2025.
  \item \emph{Sadamori Kojaku}*\footnote{Equal contribution}, Robert Mahari*, Sandro Claudio Lera, Esteban Moro, Alex Pentland, Yong-Yeol Ahn. Community-centric modeling of citation dynamics explains collective citation patterns in science, law, and patents. arXiv preprint arXiv:2501.15552, 2025.
  \item Filipi N. Silva, \emph{Sadamori Kojaku}, Alessandro Flammini, Filippo Radicchi, and Santo Fortunato. Scale invariance and statistical significance in complex weighted networks. arXiv preprint arXiv:2510.23964, 2025.
\end{itemize}
\changepapertype

\section{Manuscript in preparation}
(BU students underlined; my name in italics)
\begin{itemize}[leftmargin=0.6in, label={\paperitem}]
  \item \uline{Yoshiaki Fujita}, \uline{Akshay Gangadhar} and \emph{Sadamori Kojaku}. On the high-order cumulative advantage in citation and collaboration networks in science.
  \item \uline{Mario Franco}, \emph{Sadamori Kojaku}, and Carlos Gershenson. Is vanilla is enough, why d we need chocolate and strawberry? A Neapolitan approach to artificial neural networks.
  \item \uline{Dheeraj Tommondru}, \uline{Xuanchi Li}, \uline{Xin Wang}, and \emph{Sadamori Kojaku}. Feature selection method based on the adversarial dismantling of feature networks
  \item \uline{Xuanchi Li} and \emph{Sadamori Kojaku}. Fast inference on the maximum entropy model for networks.
\end{itemize}
\changepapertype

\section{Peer reviewed journal papers}

\begin{itemize}[leftmargin=0.6in, label={\paperitem}]
  \item Filippo Radicchi, Filipi N. Silva, Alessandro Flammini, Santo Fortunato and \emph{Sadamori Kojaku}. Detectability threshold in weighted modular networks. Physical Review E (in press).
  \item Attila Varga, \emph{Sadamori Kojaku}, Filipi Silva. Measuring Research Interest Similarity with Transition Probabilities. Quantitative Science Studies, pages 1--40, 2025.
  \item Isabel Constantino, \emph{Sadamori Kojaku}, Santo Fortunato, and Yong-Yeol Ahn. Representing the Disciplinary Structure of Physics: A Comparative Evaluation of Graph and Text Embedding Methods. Quantitative Science Studies, volume 6, pages 263--280, 2025.
  \item Yu Tian, \emph{Sadamori Kojaku}, Hiroki Sayama, and Renaud Lambiotte. Matrix-Weighted Networks for Modeling Multidimensional Dynamics: Theoretical Foundations and Applications to Network Coherence. Physical Review Letters, volume 134, number 23, pages 237401, 2025.
  \item \emph{Sadamori Kojaku}, Filippo Radicchi, Yong-Yeol Ahn, and Santo Fortunato. Network community detection via neural embeddings. Nature Communications. 15, no. 1 (2024): 9446.

  \item Bianka Kovács, \emph{Sadamori Kojaku}, Gergely Palla, Santo Fortunato. Iterative embedding and reweighting of complex networks reveals community structure. \textit{Scientific Reports}, 17184, 2024
  \item Dakota Murray, Jisung Yoon, \emph{Sadamori Kojaku}, Rodrigo Costas, Woo-Sung Jung, Sta\v{s}a Milojevi\'{c}, and Yong-Yeol Ahn. Unsupervised embedding of trajectories captures the latent structure of mobility. \textit{PNAS}, 2023
  \item \emph{Sadamori Kojaku}, Giacomo Livan, and Naoki Masuda. Detecting anomalous citation groups in journal networks. \textit{Scientific Reports}, 11, 14524, 2021 (2-year IF: 3.998)
  \item \emph{Sadamori Kojaku}, Laurent H\'ebert-Dufresne, Enys Mones, Sune Lehmann, and Yong-Yeol Ahn. The effectiveness of backward contact tracing in networks. \textit{Nature Physics}, 1745-2481, 2021 (2-year IF: 19.256)
  \item \emph{Sadamori Kojaku} and Naoki Masuda. Constructing networks by filtering correlation matrices: A null model approach. \textit{Proceedings of the Royal Society A}, 475, 2231, 2019 (2-year IF: 2.741)
  \item \emph{Sadamori Kojaku}, Mengqiao Xu, Haoxiang Xia, and Naoki Masuda. Multiscale core-periphery structure in a global liner shipping network. \textit{Scientific Reports}, 9, 404, 2019 (2-year IF: 3.998)
  \item \emph{Sadamori Kojaku}, Giulio Cimini, Guido Caldarelli, and Naoki Masuda. Structural changes in the interbank market across the financial crisis from multiple core-periphery analysis. \textit{Journal of Network Theory in Finance}, 4, 33-51, 2018
  \item Naoki Masuda, \emph{Sadamori Kojaku}, and Yukie Sano. A configuration model for correlation matrices. \textit{Physical Review E}, 98, 012312, 2018 (2-year IF: 2.296)
  \item \emph{Sadamori Kojaku} and Naoki Masuda. A generalised significance test for individual communities in networks. \textit{Scientific Reports}, 8, 7351, 2018 (2-year IF: 3.998)
  \item \emph{Sadamori Kojaku} and Naoki Masuda. Core-periphery structure requires something else in the network. \textit{New Journal of Physics}, 20, 043012, 2018 (2-year IF: 3.539)
  \item \emph{Sadamori Kojaku} and Naoki Masuda. Finding multiple core-periphery pairs in networks. \textit{Physical Review E}, 96, 052313, 2017 (2-year IF: 2.296)
  \item \emph{Sadamori Kojaku}, Ichigaku Takigawa, Mineichi Kudo, and Hideyuki Imai. Dense core model for cohesive subgraph discovery. \textit{Social Networks}, 44, 143-152, 2016 (2-year IF: 2.376)
  \item \emph{Sadamori Kojaku}, Kota Watanabe, Hajime Igarashi. Rational Forgettable Profit Sharing Reinforcement Learning. IEEJ, 3, 448-454, 2012. (original in Japanese).
  %\emph{幸若完壮}, 渡辺 浩太, 五十嵐 一. 合理的な忘却型Profit Sharing強化学習法. \textit{電気学会論文誌C} (電子・情報・システム部門誌), 3, 448-454, 2012
  \item \emph{Sadamori Koujaku}, Kota Watanabe, and Hajime Igarashi. Adaptive profit sharing reinforcement learning for dynamic environment. 10th International Conference on Machine Learning and Applications and Workshops. Hawaii, the United States, 2011
\end{itemize}
\changepapertype

\section{Peer reviewed full conference papers}

\begin{itemize}[leftmargin=0.6in, label={\paperitem}]
  \item \uline{Xuanchi Li}, \uline{Xin Wang}, \emph{Sadamori Kojaku}. Fast Unbiased Sampling of Networks with Given Expected Degrees and Strengths. Proceedings | Complex Networks and their Applications 2025 (Vestal, New York).
  \item \uline{Yeunkyung Cho}, \uline{Sarah Escotto-Rodríguez}, \emph{Sadamori Kojaku}. Emotion contagion through emotion management. Proceedings | Complex Networks and their Applications 2025 (Vestal, New York).
  \item \uline{Xin Wang}, Stephanie Tulk Jesso, \emph{Sadamori Kojaku}, David M Neyens, Min Sun Kim. VizTrust: A Visual Analytics Tool for Capturing User Trust Dynamics in Human-AI Communication. Proceedings of the Extended Abstracts of the CHI Conference on Human Factors in Computing Systems, pages 1--10, 2025.
  \item Rachith Aiyappa, \uline{Xin Wang}, Munjung Kim, Ozgur Can Seckin, Yong-Yeol Ahn, \emph{Sadamori Kojaku}. Implicit degree bias in the link prediction task. Forty-second International Conference on Machine Learning, 2025.
  \item \uline{Artin Tonekaboni}, \uline{Xin Wang}, \emph{Sadamori Kojaku}, Luis M Rocha. BingAster at\#SMM4H-HeaRD 2025: Identifying Dementia Caregivers on Twitter Using Prompt-Based LLMs and Cognitive Distortion Patterns. 2025.
  \item \emph{Sadamori Kojaku}, Jisung Yoon, Isabel Constantino, and Yong-Yeol Ahn. Residual2Vec: Debiasing graph embedding with random graphs. NeurIPS, 2021 (acceptance rate 26\%).
  \item Xia Cui, \emph{Sadamori Kojaku}, Naoki Masuda, and Danushka Bollegala. Solving feature spareness in text classification using core-periphery decomposition. In Proceedings of the 7th Joint Conference on Lexical and Computational Semantics, pages 225-264, ACL, New Orleans, USA, June 5-6, 2018.
  \item \emph{Sadamori Koujaku}, Mineichi Kudo, Ichigaku Takigawa, and Hideyuki Imai. Community change detection in dynamic networks in noisy environment. The 24th International Conference on World Wide Web, Florence, Italy, May 18 - 22, 2015.
  \item \emph{Sadamori Koujaku}, Mineichi Kudo, Ichigaku Takigawa, and Hideyuki Imai. Structural change point detection for social networks.  The World Congress on Engineering, London, the United Kingdom, July 3-5, 2013.
\end{itemize}
\changepapertype

\section{Commentary}

\begin{itemize}[leftmargin=0.6in, label={\paperitem}]
  \item \emph{Sadamori Kojaku}, New developments in network science through embedding methods. Special Issue "Frontiers of Complex Network Research" Journal of the Society of Instrument and Control Engineers, 65, 5, 185-191 (2021)
  %\item \underline{幸若完壮}, 埋め込み法が拓くネットワーク科学の新展開. 特集号「複雑ネットワーク研究の最前線」 システム制御情報学会論文誌, 65, 5, 185-191 (2021)
\end{itemize}
\changepapertype

\section{Invited Talks}

\begin{itemize}[leftmargin=0.6in, label={\paperitem}]
  \item \emph{Sadamori Kojaku}. Keynote Speaker. 17th International Conference on Complex Networks (CompleNet 2026), Zaragoza, Spain, May 4-8, 2026.
  \item Invited Member. Project: 複雑ネットワークにおける高次構造のモデリングと遠隔ノード間の影響伝播・頑健性の解明 (Modeling Higher-Order Structures in Complex Networks and Elucidating Influence Propagation and Robustness Between Remote Nodes) (PI: Yasuko Yamano). 2026.
  \item \emph{Sadamori Kojaku}. Neural embeddings unveil simplicity in complex systems. American Physical Society March Meeting, Minneapolis, MN, USA, March 4 - 8, 2024.
  \item \emph{Sadamori Kojaku}. Distilling rich but crude scholarly data using representation learning, IUNI Lunch Colloquium: Science of Science and Networks, Indiana, USA, October 28, 2022.
  \item \emph{Sadamori Kojaku}, Jisung Yoon, Isabel Constantino, and Yong-Yeol Ahn. Residual2Vec: Debiasing graph embedding with random graphs. Network Inequality, International School and Conference on Network Science (NetSci) 2022, Shanghai, China, July 21, 2022.
  \item \emph{Sadamori Kojaku}. Algorithms for detecting network cores and their applications. Network Science Seminar, Institute of Statistical Mathematics, Japan, August 28-30, 2019.
  %\item \emph{幸若完壮}. ネットワークコアの検出アルゴリズムとその応用. ネットワーク科学セミナー, 統計数理研究所, Japan, August 28-30, 2019.
  \item \emph{Sadamori Kojaku}, Laurent Hébert-Dufresne, Enys Mones, Sune Lehmann, and Yong-Yeol Ahn. The effectiveness of backward contact tracing in networks. The State University of New York at Buffalo, June 4, 2021.
\end{itemize}
\changepapertype

\section{Oral Presentations \& Poster Presentations (Refereed)}
(BU students underlined; my name in italics)

\subsubsection*{Oral Presentations}
\begin{itemize}[leftmargin=0.6in, label={\paperitem}]
\item \uline{Yoshiaki Fujita}, \uline{Akshay Gangadhar} and \emph{Sadamori Kojaku}. On the high-order cumulative advantage in citation and collaboration networks in science. NetSci, Maastricht, NERCCS, April 2025.
\item \emph{Sadamori Kojaku}*\footnote{* refers to the presenter}, Filippo Radicchi, Yong-Yeol Ahn, and Santo Fortunato. Network community detection via neural embeddings. CompleNet, Casa José de Alencar, Brazil, April 2025.
\item Rachith Aiyappa, \uline{Xin Wang}, Munjung Kim, Ozgur Can Seckin, Jisung Yoon, Yong-Yeol Ahn, \emph{Sadamori Kojaku}. Implicit degree bias in the link prediction task. CompleNet, Casa José de Alencar, Brazil, April 2025.
\item Rachith Aiyappa, \uline{Xin Wang}, Munjung Kim, Ozgur Can Seckin, Jisung Yoon, Yong-Yeol Ahn, \emph{Sadamori Kojaku}. Implicit degree bias in the link prediction task. NetSci, Maastricht, the Netherlands, June 2025.
\item \uline{Xuanchi Li}, \uline{Xin Wang}, \emph{Sadamori Kojaku}. Near-linaer time algorithm for the configuration models for networks. CompleNet, Casa José de Alencar, Brazil, April 2025.
\item \uline{Xuanchi Li}, \uline{Xin Wang}, \emph{Sadamori Kojaku}. Near-linaer time algorithm for the configuration models for networks. NetSci, Maastricht, the Netherlands, June 2025.
\item \emph{Sadamori Kojaku}, Robert Mahari, Sandro Lera, Esteban Moro, Alex Pentland, Yong-Yeol Ahn. Uncovering the universal dynamics of citation systems: From science of science to law of law and patterns of patents. NERCCS, Clarkson, NY, USA, March 20 - 22, 2024,
\item \emph{Sadamori Kojaku}, *Robert Mahari, Sandro Lera, Esteban Moro, Alex Pentland, Yong-Yeol Ahn. Uncovering the universal dynamics of citation systems: From science of science to law of law and patterns of patents. International School and Conference on Network Science (NetSci), Vienna, Austria, Jul 12 - 14, 2023.
\item \emph{Sadamori Kojaku}, *Robert Mahari, Sandro Lera, Esteban Moro, Alex Pentland, Yong-Yeol Ahn. Uncovering the universal dynamics of citation systems: From science of science to law of law and patterns of patents. ICSSI, Chicago, IL, USA, June 26 - 29, 2023.
\item \emph{Sadamori Kojaku}, Filippo Radicchi, Yong-Yeol Ahn, and Santo Fortunato. Network community detection via neural embeddings. International School and Conference on Network Science (NetSci), Vienna, Austria, Jul 12 - 14, 2023.
\item *\emph{Sadamori Kojaku}, Clara Boothby, Filipi Nascimento Silva, Attila Varga, Xiaoran Yan, Sta\v{s}a Milojevi\'{c}, Alessandro Flammini, Filippo Menczer, and Yong-Yeol Ahn. Maping Scientific Foraging. ICSSI. Washington D.C., USA, 6-9 June 2022.
\item *\emph{Sadamori Kojaku}, Xiaoran Yan, Jisung Yoon, Filipi N. Silva, Vincent Lariviere, and Yong-Yeol Ahn. DisamBERT: author name disambiguation with BERT. ICSSI. Washington D.C., USA, 6-9 June 2022.
\item *\emph{Sadamori Kojaku}, Laurent H\'ebert-Dufresne, Enys Mones, Sune Lehmann, and Yong-Yeol Ahn. The effectiveness of backward contact tracing in networks. International School and Conference on Network Science (NetSci). Virtual, 05-10 July 2021.
\item *\emph{Sadamori Kojaku}, Jisung Yoon, and Yong-Yeol Ahn. Residual2Vec: A null model approach for graph embedding. International School and Conference on Network Science (NetSci). Virtual, 05-10 July 2021.
\item Dakota Murray, *Jisung Yoon, \emph{Sadamori Kojaku}, Rodrigo Costas, Woo-Sung Jung, Sta\v{s}a Milojevi\'{c}, and Yong-Yeol Ahn. Unsupervised embedding of trajectories captures the latent structure of mobility. International School and Conference on Network Science (NetSci). Virtual, 05-10 July 2021.
\item *\emph{Sadamori Kojaku}, Attila Varga, Xiaoran Yan, Filipi N. Silva, Staša Milojević, Alessandro Flammini, and Yong-Yeol Ahn. The landscape of the COVID-19 research: A neural embedding approach. International School and Conference on Network Science (NetSci). Rome, Italy, 17-25 September 2020.
\item *\emph{Sadamori Kojaku}, Giacomo Livan, and Naoki Masuda. Detecting citation cartels in journal networks. International School and Conference on Network Science (NetSci). Rome, Italy, 17-25 September 2020.
\item *\emph{Sadamori Kojaku}, Giulio Cimini, Guido Caldarelli, and Naoki Masuda. Structural changes in the interbank market across the financial crisis from multiple core-periphery analysis. International School and Conference on Network Science (NetSci). Vermont, U.S., May 26-31 2019.
\item *\emph{Sadamori Kojaku} and Naoki Masuda. Core-periphery structure in degree-heterogeneous networks. International School and Conference on Network Science (NetSci-X). Hangzhou, China 2018.
\item *\emph{Sadamori Kojaku} and Naoki Masuda. Finding multiple core-periphery structure with random walks. 5th International Workshop on Complex Networks and their Applications. Milan, Italy November 30-December 2 2016.
\item *Keigo Kimura, Mineichi Kudo, Lu Sun, and \emph{Sadamori Kojaku}. Fast random k-labelsets for large-scale multi-label classification. 23rd International Conference on Pattern Recognition. Cancun, Mexico December 4-8 2016.
\end{itemize}

\subsubsection*{Poster Presentations}
\begin{itemize}[leftmargin=0.6in, label={\paperitem}]
\item *Rachith Aiyappa, \uline{Xin Wang}, Munjung Kim, Ozgur Can Seckin, Jisung Yoon, Yong-Yeol Ahn, \emph{Sadamori Kojaku}. Implicit degree bias in the link prediction task. Forty-second International Conference on Machine Learning (ICML), Vienna, Austria, July 21 - 27, 2025.
\item \uline{Yoshiaki Fujita}, \uline{Akshay Gangadhar} and \emph{Sadamori Kojaku}. On the high-order cumulative advantage in citation and collaboration networks in science. NetSci, Maastricht, the Netherlands, June 2025.
\item *\underline{Xin Wang} and \emph{Sadamori Kojaku}. User Trust Modeling in Conversational User Interface Based on Word Embedding Bias. The ACM Conference on Conversational User Interfaces. July 8 - 10, 2024.
\item *Govind Gandhi, Yong-Yeol Ahn, and \emph{Sadamori Kojaku}. Self-Supervised Modularity Maximization using graph embeddings for clustering. International School and Conference on Network Science (NetSci), Quebec, Canada, June 16 - 21, 2024.
\item *\underline{Xin Wang} and \emph{Sadamori Kojaku}. Analyzing Patient Reviews on Google Map Hospital Profiles through Neural Embedding and Network Modeling. NERCCS, Clarkson, NY, USA, March 20 - 22, 2024.
\item Ashutosh Tiwari, *\emph{Sadamori Kojaku} and Yong-Yeol Ahn. A biased contrastive learning debiases graph neural networks. International School and Conference on Network Science (NetSci). Vienna, Austria, Jul 12 - 14, 2023.
\item *\emph{Sadamori Kojaku}, Clara Boothby, Filipi Nascimento Silva, Attila Varga, Xiaoran Yan, Sta\v{s}a Milojevi\'{c}, Alessandro Flammini, Filippo Menczer, and Yong-Yeol Ahn. Understanding the landscape of COVID-19 research by using neural embedding. ICSSI. Chicago,  IL, USA, June 26 - 29, 2023.
\item *Munjung Kim, \emph{Sadamori Kojaku}, and Yong-Yeol Ahn. Quantifying disruptiveness using a neural embedding method. ICSSI. Chicago, IL, USA, June 26 - 29, 2023.
\item *\emph{Sadamori Kojaku} and Naoki Masuda. Constructing networks from correlation matrices: An application to economical data. Threshold Networks. Nottingham, UK, July 22-24, 2019.
\item *\emph{Sadamori Kojaku} and Naoki Masuda. A generalised significance test for individual communities in networks. International School and Conference on Network Science (NetSci). Paris, France, June 11–15, 2018.
\item *\emph{Sadamori Kojaku} and Naoki Masuda. Multi-scale organisation of core-periphery structure in networks. 1st Latin American Conference on Complex Networks. Puebla, Mexico, September 25-29, 2017.
\item *\emph{Sadamori Kojaku} and Naoki Masuda. Core-periphery structure of networks: Consideration for random heterogeneous networks. International School and Conference on Network Science (NetSci). Indianapolis, Indiana, USA, 2017.
\item *\emph{Sadamori Kojaku} and Naoki Masuda. An extension of modularity for finding multiple core/periphery structure in networks. International School and Conference on Network Science (NetSci-X). Tel Aviv, Israel, January 15-18, 2017.
\end{itemize}
\changepapertype

\section{Grants}

\begin{itemize}[leftmargin=0.6in, label={\paperitem}]
  \item PI: \emph{Sadamori Kojaku}. Co-PI: Bryan Acton. Project: TEAMS: Team Engineering for Industrial AI Systems. Funded by SUNY-IBM AI Research Alliance. Feb. 2026 -- Jul. 2027. \$100,000 USD.
  \item PI: \emph{Sadamori Kojaku}, Giulio Cimini. Co-PI: Guido Caldarelli, Daigo Uemoto, and Takashi Kamihigashi. Project: Correlation-based reconstruction of financial networks for systemic risk control. Funded by JSPS Bilateral Exchange Program. 2020. 8,000,000 JPY (withdrawn due to transition to a U.S. institution)
\end{itemize}

%\subsubsection*{Submitted}
%
%\begin{itemize}[leftmargin=0.6in, label={\paperitem}]
%  \item PI: \emph{Sadamori Kojaku}. III: Small: Cross-domain Social Entity Resolution through Text-to-Text Large Language Models. Submitted to NSF CISE-SMALL. 500k USD.
%  \item PI: Luis Rocha at Binghamton University. Co-PI: Trueblood Miller, Wendy Renee, Katy Borner, Bruce W. Herr II, Felipe Xavier Costa, Xuan Wang, Elizabeth Record, Sadamori Kojaku. MyAura. Submitted to Ho Foundation.
%\end{itemize}
%
%\subsubsection*{In preparation}
%\begin{itemize}[leftmargin=0.6in, label={\paperitem}]
%  \item PI: \emph{Sadamori Kojaku}. NSF CAREER Program. Universal representation of intellectual mobility in science, technology, and society. 2025. Participated in Commit-to-Submit (C2S) program in Binghamton University in 2024.
%  \item PI: Bryan Acton, \emph{Sadamori Kojaku}, and Rory Eckardt. Transdisciplinary Areas of Excellence in Data Science.
%  \item PI: \emph{Sadamori Kojaku}, Bryan Acton, and Rory Eckardt. Science of Organization. NSF.
%  \item PI: \emph{Sadamori Kojaku}. Co-PI: Dakota Murray at Northeastern and Yong-Yeol Ahn at Indiana University. HNDS-R: Charting intellectual space: an infrastructure for a unified, interpretable, cross-domain embedding of intellectual mobility. NSF HNDS-R.
%  \item PI: \emph{Sadamori Kojaku}. Co-PI: Naoki Masuda at University at Buffalo and Hiroki Sayama at Binghamton University. Collaborative Research: ATD: Charting threat landscape with an interpretable multi-modal embedding of spatiotemporal data. NSF MFAI.
%\end{itemize}
%
%\subsubsection*{Rejected}
%
%\begin{itemize}[leftmargin=0.6in, label={\paperitem}]
%  \item PI: \emph{Sadamori Kojaku}. Co-PI: Dakota Murray at Northeastern and Yong-Yeol Ahn at Indiana University. HNDS-I: Charting intellectual space: an infrastructure for a unified, interpretable, cross-domain embedding of intellectual mobility. Submitted to NSF HNDS-I. 819k USD (319k USD for Binghamton).
%  \item PI: \emph{Sadamori Kojaku}. Co-PI: Naoki Masuda at University at Buffalo and Hiroki Sayama at Binghamton University. Collaborative Research: ATD: Charting threat landscape with an interpretable multi-modal embedding of spatiotemporal data. Submitted to NSF ATD. 264k USD (190k USD for Binghamton)
%  \item PI: Co-PI: Luis Rocha at Binghamton University and \emph{Sadamori Kojaku}. Complex Network Exploration through Backbone-Enhanced Techniques with Intelligent Support. Submitted to NSF-ANR. (600k USD for Binghamton)
%\end{itemize}

\changepapertype

\section{Patents}
\begin{itemize}[leftmargin=0.6in, label={\paperitem}]
  \item \emph{Sadamori Kojaku} and Takashi Kamihigashi. Academic paper reviewer search device, reviewer search method, and reviewer search program. 2024 [7470369. Japan Patent Office]
  \item \emph{Sadamori Kojaku} and Takayuki Osogami. Prediction method, prediction system and program. 2013 [Patent No: 9087294. Japan Patent Office].
\end{itemize}

\section{Honors}
\begin{itemize}[leftmargin=0.6in]
  \item Best Presentation Award. The International Conference on Complex Networks (CompleNet). \hfill 2025
  \item Art of Science Competition Finalist, Judges Choice. Binghamton University [14/85 entries]. \hfill 2024
  \item Outstanding Faculty Award from Students with Disabilities. Binghamton University. \hfill 2024
  \item Best Contribution on Financial Networks Award. International School and Conference on Network Science (NetSci-X).  [1/58 presenters]. \hfill 2017
  \item Dean Award. \textit{Graduate School of Information Science and Technology, Hokkaido Univ.}\hfill 2015
  \item Best Student Award. \textit{The World Congress on Engineering} \hfill 2013
\end{itemize}

\section{Teaching}
\begin{description}[labelwidth=3cm,font=\normalfont]
  \item[SSIE 419/519] Applied Soft Computing. Instructor (Spring 2024). Binghamton University
  \item[SSIE 641] Advanced Topics in Network Science. Instructor (Fall 2023). Binghamton University
  \item[BL-INFO-I590] Data Visualization. Instructor (Fall 2023). Indiana University.
\end{description}

%\section{Student Opinion Of Teaching (SOOT) Survey Percentages}
%
%\begin{table}[h!]
%  \centering
%  \begin{adjustbox}{max width=\textwidth}
%\begin{tabular}{p{3cm}p{5cm}p{1cm}p{1cm}p{1cm}p{1cm}p{1cm}p{1cm}p{1cm}p{1cm}p{1cm}}
%\toprule
%Course & Term & Q1 (\%) & Q2 (\%) & Q3 (\%) & Q4 (\%) & Q5 (\%) & Q6 (\%) & Q7 (\%) & Q8 (\%) & Q9 (\%) \\
%\midrule
%SSIE 519/419 & Spring 24 (6 responses) & 83 & 83 & 83 & 50 & 67 & 88 & 88 & 67 & 67 \\
%SSIE 641 & Fall 24 (8 responses) & 88 & 88 & 75 & 50 & 88 & 100 & 100 & 88 & 88 \\
%SSIE 519/419 & Spring 24 (3 responses) & 100 & 100 & 100 & 100 & 100 & 100 & 100 & 100 & 100 \\
%SSIE 641 & Fall 23 (5 responses) & 60 & 60 & 40 & 40 & 60 & 100 & 80 & 60 & 80 \\
%\bottomrule
%\end{tabular}
%\end{adjustbox}
%\caption{Percentage of "High" and "Very High" responses for each question in the Student Opinion Of Teaching (SOOT) Survey}
%\end{table}

\section{PhD Student Guidance Committee}
(As Advisor or Co-Advisor)

\begin{description}[labelwidth=3cm,font=\normalfont]
  \item[Chand Sahil Mansuri] (SS), Expected to Graduate in 2027.
  \item[Xin (Vision) Wang] (SS), Expected to Graduate in 2027.
  \item[Luke Netto] (SS), Expected to Graduate in 2028.
  \item[Xuanchi Li] (SS), Expected to Graduate in 2027.
  \item[Hayford Adjavor] (SS), Expected to Graduate in 2026.
\end{description}


\section{MS Student Guidance Committee}
(As a commitee member)

\begin{description}[labelwidth=3cm,font=\normalfont]
  \item[Miriam Flores] (SS). The Impact of Autonomous Vehicles on Traffic Flow. Expected to Graduate in 2024.
\end{description}

\section{International and Professional Services}

\begin{itemize}[leftmargin=0.6in, label={\paperitem}]
  \item Chair of Rep4CS, a satellite workshop on Representation learning for complex systems, as a part of Conference on Complex Systems 2024 at Exeter in UK.
  \item Organizer of NetSci-X 2020, International School and Conference on Network Science in Tokyo. 2020.
  \item Program Committee Member in International School and Conference on Network Science (NetSci) (2019, 2020, 2021, 2022, 2023).
  \item Referee for Nature Human Behavior; Nature Communications; Scientific Reports; Journal of Complex Networks; Journal of Computational Social Science; PLOS ONE; ECAI
\end{itemize}

\changepapertype


\section{Service within the University}

\begin{itemize}[leftmargin=0.6in, label={\paperitem}]
  \item Talk at DataViz Community Presentations and Networking. Gravity of Ideas: Mapping Science at BU. Spring 2025.
  \item Talk at DataViz Community Presentations and Networking. Unsupervised embedding of trajectories captures the latent structure of scientific migration. Fall 2024.
  \item Talk at Data Salon. Machine learns simplicity from complexity. Nov 17, 2023.
  \item Talk at DataViz Pi Day 2024 Presentations and Networking. A Landscape of All Sciences: Neural Networks of Over 400M Publications. March 14, 2023.
\end{itemize}
%    \resumeSubHeadingListStart
%      \resumeProjectHeading
%          {\textbf{Gitlytics} $|$ \emph{Python, Flask, React, PostgreSQL, Docker}}{June 2020 -- Present}
%          \resumeItemListStart
%            \resumeItem{Developed a full-stack web application using with Flask serving a REST API with React as the frontend}
%            \resumeItem{Implemented GitHub OAuth to get data from user’s repositories}
%            \resumeItem{Visualized GitHub data to show collaboration}
%            \resumeItem{Used Celery and Redis for asynchronous tasks}
%          \resumeItemListEnd
%      \resumeProjectHeading
%          {\textbf{Simple Paintball} $|$ \emph{Spigot API, Java, Maven, TravisCI, Git}}{May 2018 -- May 2020}
%          \resumeItemListStart
%            \resumeItem{Developed a Minecraft server plugin to entertain kids during free time for a previous job}
%            \resumeItem{Published plugin to websites gaining 2K+ downloads and an average 4.5/5-star review}
%            \resumeItem{Implemented continuous delivery using TravisCI to build the plugin upon new a release}
%            \resumeItem{Collaborated with Minecraft server administrators to suggest features and get feedback about the plugin}
%          \resumeItemListEnd
%    \resumeSubHeadingListEnd



%
%-----------PROGRAMMING SKILLS-----------
%-------------------------------------------
\end{document}


